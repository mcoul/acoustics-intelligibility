A lo largo del trabajo, se incorporaron los conceptos base para poder diseñar una sala de conferencias que cumpla con unos parámetros determinados.\\

En la primera parte, se puede ver que el volumen del recinto es primordial para analizar la distribución de los modos a propagarse dentro del mismo; esta distribución repercute directamente en el tiempo de reverberación que tendrá la sala diseñada.\\

En la segunda parte, luego de definir los tipos de materiales que revestirían a la sala, se observó que el tiempo de reverberación calculado no se encontraba dentro de las bandas de tolerancia del tiempo óptimo previamente calculado. Luego de acondicionar el recinto con materiales fonoabsorbentes, se corroboró que el tiempo de reverberación disminuyó y se pudo así cumplir con un valor aceptable.\\

A partir del cálculo de tiempo de reverberación, se pudo desprender el valor del nivel de inteligibilidad, que fue el objetivo final a cumplir en esta experiencia.\\

Junto con la expresión de inteligibilidad, se analizó qué pasa con este parámetro para distancias menores a la llamada \emph{distancia crítica} multiplicada por un factor y para distancias mayores a ella. Se observó que para oyentes situados a distancias menores que la $D_C$, la fuente se escucharía demasiado alta, por lo que deben colocarse a una distancia mayor que este valor, para poder tener la condición de campo reverberante.



%%%%%%%%% OBSERVACIONES PEOLAS

%\begin{itemize}
%	\item Cuanto menor sea el TR, menor será el $\%AL_{Cons}$, es decir, mayor inteligibilidad;
%	\item Cuanto más cerca esté situado el receptor de la fuente sonora, mejor será la relación señal/ruido (mayor valor de LD-LR);
%	\item El valor de $\%AL_{Cons}$ va aumentando a medida que el receptor se aleja de la fuente, hasta una distancia: $r = 3.16\,D_c$. Para distancias $r > 3.16\,D_c$, equivalentes a (LD - LR) < -10 dB, el valor de $\%AL_{Cons}$ tiende a ser constante. Ello significa que, a partir de ese punto, la inteligibilidad de la palabra ya no empeora con el aumento de la distancia;
%	\item Otro factor no mencionado hasta el momento, pero que contribuye sustancialmente a la pérdida de inteligibilidad, es el ruido de fondo de la sala. Como criterio de diseño, se puede considerar que su efecto es despreciable cuando el correspondiente nivel de ruido de fondo está, como mínimo, 12 dB por debajo del nivel de la señal.
%\end{itemize}


%Criterios de diseño:
%
%\begin{itemize}
%	\item Los cálculos deben hacerse, como mínimo, para la mayor distancia entre la fuente y el receptor (distancia hasta la última fila del recinto);
%	\item En cuanto a la variación con la frecuencia, una opción es la de trabajar con valores de la banda de octava centrada en f = \SI{1}{\k\Hz}, por tratarse de una de las bandas de contribución a la inteligibilidad de la palabra, y porque es la media logarítmica del espectro de voz humana [\SI{100}{\Hz};\SI{1}{k\Hz};\SI{10}{k\Hz}];
%	\item Pero habitualmente el $\%AL_{Cons}$ se calcula en la banda de los \SI{2}{k\Hz}, por tratarse de la banda de máxima contribución a la inteligibilidad de la palabra.
%\end{itemize}


%De modo que para el cálculo de R podemos utilizar los valores de:
%
%\begin{itemize}
%	\item $A_{2kHz}$: área equivalente de absorción sonora para la octava de 2 kHz, en m2;
%	\item $\alpha_{2kHz}$: coeficiente de absorción sonora para la octava de 2 kHz, adimensional;
%	\item $T_{60}$: tiempo de reverberación de la banda de octava de 2 kHz, en s
%\end{itemize}