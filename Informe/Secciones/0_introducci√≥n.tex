	\subsection{Objetivos}
	
	El objetivo este trabajo es conseguir condiciones acústicas adecuadas para una buena inteligibilidad de la palabra en una sala de conferencias. Se debe lograr que el porcentaje de palabras correctamente interpretadas por el oyente sea mayor que el \texttt{90\%}.\\
	
	Partiendo de datos conocidos (el largo y ancho del local), se desea determinar los siguientes parámetros:
	
	\begin{itemize}
		\item Volumen del recinto;
		\item Tiempo de reverberación del recinto;
		\item Distancia crítica;
		\item Inteligibilidad.
	\end{itemize}
		
%	\subsection{Especificaciones}
%	
%	Para cumplir con el porcentaje de inteligibilidad requerido, es necesario atender dos aspectos fundamentales:
%
%	\begin{itemize}
%		\item El aislamiento acústico que brinde la envolvente del recinto, para protegerlo del ruido exterior y evitar que interfiera con las condiciones de audición exigidas por la actividad a desarrollar en él;
%		\item El acondicionamiento acústico interior, adecuando la sala al uso al que estará destinada (dimensiones, forma, materiales, etc.)
%	\end{itemize}
%
%	En este proyecto consideraremos que el aislamiento acústico ya ha sido calculado adecuadamente.