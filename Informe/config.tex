\input{.command.tex}
% En el siguiente archivo se configuran las variables del trabajo práctico
%% \providecommand es similar a \newcommnad, salvo que el primero ante un 
%% conflicto en la compilación, es ignorado.

% Al comienzo de un TP se debe modificar los argumentos de los comandos

\providecommand{\myTitle}{Trabajo prático de laboratorio} 
\providecommand{\mySubtitle}{Ensayos acústicos de pantallas}

\providecommand{\mySubject}{Acústica (86.57)}
\providecommand{\myKeywords}{UBA, Ingeniería, Control}

\providecommand{\myAuthorSurname}{Lean Cole}
\providecommand{\myTimePeriod}{Año 2019 - 2\textsuperscript{do} Cuatrimestre}

% No es necesario modificar este %%%%%%%%%%%%%%
\providecommand{\myHeaderLogo}{header_fiuba}
%%%%%%%%%%%%%%%%%%%%%%%%%%%%%%%%%%%%%%%%%%%%%%%%

% Si se utilizan listings, definir el lenguaje aquí
\providecommand{\myLanguage}{matlab} 
% Crear los integrantes del TP con el comando \PutMember donde
%%		1) Apellido, Nombre
%%		2) Número de Padrón
%%		3) E-Mail
\providecommand{\MembersOnCover}[0]
{		
		\PutMember{Lean Cole, Micaela}{96364}{leancolem@gmail.com}
}

\providecommand{\myGroupNumber}{25}


\Pagebreaktrue		% Setea si hay un salto de página en la carátula
\Indextrue
\Siunitxtrue			% Si quiero utilizar el paquete, \siunixtrue. Si no \siunixfalse
\Todonotestrue		% Habilita/Deshabilita las To-Do Notes y las funciones \unsure, \change, \info, \improvement y \thiswillnotshow.
\Listingstrue
\Keywordsfalse
\Putgroupfalse		% Habilita/Deshabilita el \myGroup en los headers
\Videofalse
